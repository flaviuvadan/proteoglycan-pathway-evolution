\documentclass{article}

\usepackage[doublespacing]{setspace}
\usepackage{geometry}
\usepackage{verbatim}

\title{Mutation ratios of proteoglycan synthesis genes in organisms with specific bone classifications}
\author{Flaviu Vadan, IanMcQuillan, Brian F. Eames}
\date{January 11th, 2020}

\begin{document}
\maketitle

\section{Introduction}
Proteoglycans are a variety of carbohydrate-coated proteins. Proteoglycans can be found in the extracellular matrix, where they are involved in synthesizing articular cartilage, and intracellularly where they have been recently identified to function in processes such as growth factor signalling. The loss of extracellular proteoglycans is associated with debilitating  diseases such as osteoarthritis and recent findings that suggest proteoglycan to be located intracellularly have helped with gaining a better understanding of the consequences of proteoglycan loss. 

Proteoglycans are composed of a corep rotein onto which different carbohydrates are attached. The function of the proteoglycan and its tissue presence is dictated by the types of surface-attached carbohydrates. For example,  chondroitin sulfate has repeating disaccharides of glucoronic acid and acetylgalactosamine whereas heparan sulfate has glucoronic acid and acetylgalactosamine repeats.

\section{Motivation}
As already mentioned, one of the core motivators for studying the synthesis and origins of proteoglycan is its association with diseases. Further, according to recent findings that suggest proteoglycan to be secreted intracellularly, we have additional motivation for decoding whether intracellular proteoglycan synthesis disruptions are associated with disease as well. A better understanding of proteoglycan synthesis, as it occurs in different organisms, might help inform test organism choices for osteoarthritis research such as studying how organisms react to the onset of osteoarthritis or how they react to potential treatments.

\section{Approach}
A potential contributor to our overall knowledge of proteoglycan synthesis is the understanding of the origins of proteoglycan synthesis pathway and how the genes in the pathway compare between organisms from different classifications. Since proteoglycan synthesis is associated with bone and cartilage development, organisms can be classified according to bone presence - there are organisms that synthesize bone and cartilage, only cartilage, or neither of them - and environment - organisms from different environments, such as terrestrial versus aquatic, may exhibit different synthesis patterns as a consequence of genetic differences. 

\section{Objective}
As a consequence, we decided to test the hypothesis that genes in the proteoglycan synthesis pathway exhibit higher non-synonymous vs. synonymous mutation ratios in organisms that have specific bone/cartilage traits and live in specific environments such as aquatic or terrestrial. 

\end{document}